\documentclass[a3paper,12pt,nohead]{azmoon}
\usepackage{xepersian}
\settextfont{Yas}
\setdigitfont{Yas}
\title{کلاس azmoon}
\author{مجتبی باغبان}
\linespread{1.5}
\newgeometry{left=2cm,right=2cm,top=2cm,bottom=2cm}
\begin{document}
	\maketitle
	\tableofcontents
	\section{مقدمه}
		کلاس azmoon برای تایپ راحت برگه‌ی امتحانی طراحی شده است تا کاربر هنگام طراحی سوال نیازی به طراحی سربرگ و قالب‌بندی سوالات نداشته باشد. کلاس azmoon طراحی سربرگ و قالب‌بندی سوالات را به‌صورت خودکار انجام می‌دهد.
	\section{فراخوانی کلاس}
	مانند تمام کلاس‌ها، کلاس azmoon نیز باید در ابتدای سند با دستور
	\begin{LTR}
	\begin{verbatim}
	\documentclass[options]{azmoon}
	\end{verbatim}
	\end{LTR}
	فراخوانی شود. گزینه‌هایی که می‌توانیم در فراخوانی کلاس استفاده کنیم عبارتند از:
	\subsection{اندازه برگه}
	\begin{latin}
	\begin{itemize}
		\item \verb|papersize=a3paper|
		\item \verb|papersize=a4paper| $\star$
		\item \verb|papersize=a5paper|
	\end{itemize}
	\end{latin}
	\subsection{اندازه فونت}
	\begin{latin}
		\begin{itemize}
			\item \verb|fontsize=10pt|
			\item \verb|fontsize=11pt| $\star$
			\item \verb|fontsize=12pt|
		\end{itemize}
	\end{latin}
	\subsection{نوع قالب}
	\begin{latin}
		\begin{itemize}
			\item \verb|template=fantasy| $\star$
			\item \verb|template=tabling|
			\item \verb|template=classic|
		\end{itemize}
	\end{latin}
\end{document}